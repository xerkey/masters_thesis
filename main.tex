\documentclass[12pt,a4j]{jreport}
\usepackage{graphicx}  % Overleafではdvipdfmx不要
\usepackage{otf}  % 日本語用フォントサポート
\usepackage{caption}  % キャプションの整形
\usepackage{cite}  % 引用スタイルの統一

\begin{document}

% タイトルページ
\thispagestyle{empty}
\begin{center}
修士論文/ 課題研究報告書\\% どちらか一方を消してください.
\vfill
〇〇〇〇〇題目〇〇〇〇〇\\
\vfill
〇〇著者名〇〇\\
\vfill
主指導教員  〇〇 〇〇\\
\vfill
北陸先端科学技術大学院大学\\
先端科学技術研究科\\
(〇〇〇〇)\\ %取得希望学位
\vfill
令和〇〇年〇月\\ % 学位授与年月
\vfill
\end{center}

\clearpage
\thispagestyle{plain}

% Abstract
\centerline{Abstract}
〇〇〇〇〇〇〇〇〇〇〇〇〇〇〇〇〇〇〇
\clearpage

% 目次
\tableofcontents
\clearpage

% 図目次と表目次
\listoffigures
\clearpage
\listoftables
\clearpage

% ページ番号をリセット
\setcounter{page}{1}

% 本文
\chapter{はじめに}
〇〇〇〇〇〇〇〇〇〇〇〇〇〇〇〇〇〇\cite{ref1,ref2}

\chapter{関連研究}
\section{セクション名}
\subsection{サブセクション名}
〇〇〇〇〇〇〇〇〇〇〇〇〇〇〇〇〇〇

\chapter{提案手法}
〇〇〇〇〇〇〇〇〇〇〇〇〇〇〇〇〇〇 (図 \ref{fig1})

\begin{figure}[h]
\centering
\includegraphics[width=90mm]{image/icon.png}
\caption{図のキャプション}\label{fig1}
\end{figure}

\chapter{実験・評価}
\begin{table}[h]
\centering
\begin{tabular}{r|rr}
& a & b\\ \hline
1& 0.25 & 0.33\\
2& 0.75 & 0.66\\
\end{tabular}
\caption{表のキャプション}\label{table1}
\end{table}

\chapter{おわりに}

% 参考文献
\renewcommand{\bibname}{参考文献}
\begin{thebibliography}{99}
\bibitem{ref1} 著者名.本のタイトル,出版社 (出版年)
\bibitem{ref2} 著者1 and 著者2. 論文タイトル, 学会名, pp.zz--ww (2007)
\bibitem{ref3} 著者1, 著者2, and 著者3. 論文タイトル,ジャーナル名,vol.xx, no.yy, pp.zz--ww (2003)
\end{thebibliography}

\end{document}
