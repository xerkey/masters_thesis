\centerline{Abstract}
This study aims to establish a system for extracting and sharing the knowledge of skilled practitioners in a natural context within the field of skill transfer. To achieve this goal, we propose a novel method combining dialogue collection in instructional settings with knowledge extraction support using Large Language Models (LLMs), and verify its effectiveness through empirical experiments in ballroom dancing.

The empirical study analyzed five weeks of practical data from two instructors and thirteen learners. The results showed that 76\% of instructional comments contained previously unincorporated elements, demonstrating the possibility of extracting new knowledge elements through natural interactions in daily instructional activities. Furthermore, LLMs demonstrated high accuracy (F-value: 0.928) in extracting unincorporated elements, confirming their contribution to systematic knowledge organization.

A significant finding was the discovery of the dual nature of knowledge structure in ballroom dancing: the aspect of chronological sequence of movements and the underlying theoretical knowledge. Based on this finding, we propose two classification frameworks: a four-layer structure for understanding skills and a classification based on execution layer characteristics. Drawing from these frameworks, we propose a new knowledge representation method that separately describes chronological movements in a single layer and represents underlying theoretical knowledge as purpose-oriented knowledge. This approach has potential applications not only in ballroom dancing but also in other physical skill domains with similar characteristics.
\clearpage