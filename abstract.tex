\centerline{Abstract}
技能継承は高齢化社会の重要課題である. 従来, 技能継承は主に教科書やマニュアルを通じて行われてきた. しかし, これらの文書だけでは熟練技能者が長年の経験から獲得した暗黙知や微妙なコツを十分に伝えきれないという問題がある. この問題を解決するために熟練者の暗黙知を様々な角度から抽出可能な知識構造化という手法がある. 本研究では, 大規模言語モデル(LLM)を活用し, チャット履歴を自動解析して新たな技能要素を抽出し, 知識構造化を支援するシステムを提案する. 本システムは既存の動作モデルに対し, どの時点で手順やコツを追加・修正すべきかを提案する. これにより, 熟練者の経験に基づく動作の技能を効率的に表出化し, 学習者の技能習得を促進することが期待される. 本研究では社交ダンスを例として大規模言語モデルの可能性を検証する. 
\clearpage