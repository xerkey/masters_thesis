\chapter{はじめに}
 熟練技能者の高齢化や後継者不足が深刻化する中,技能伝承が課題になっている. 労働政策研究・研修機構の調査によれば, 2020年の時点で調査対象の企業のうち技能継承を重要だと考えている企業が95\%に達している. しかしながら, 技能者の人材育成や能力開発の取り組みがうまくいっていると認識している企業は約55\%にとどまっている \cite{JILPT2020}. \\
 技能伝承を実現するためには, 熟練技能者が無意識のうちに身につけている技能やノウハウ, すなわち暗黙知を表出し, それを育成対象の人材にわかりやすく共有する必要がある.\\
 熟練技能者の暗黙知を表出する手法として, インタビューベースの方法が用いられている\cite{Onozato1998, Yashiro2021,Ogawa2011}. \\
 図\ref{fig1}が示すように...\\
 表\ref{table1}が示すように...\\

\begin{figure}[htbp]
    \centering
    \includegraphics[width=0.8\linewidth]{./image/icon.png}
    \caption{画像の説明キャプション}
    \label{fig1}
\end{figure}

\begin{table}[h]
    \centering
    \begin{tabular}{r|rr}
    & a & b\\ \hline
    1& 0.25 & 0.33\\
    2& 0.75 & 0.66\\
    \end{tabular}
    \caption{表のキャプション}
    \label{table1}
\end{table}