\chapter{はじめに}
- 技能伝承の必要性 -\\
 熟練技能者の高齢化や後継者不足が深刻化する中, 長年培われてきた技能の存続が危ぶまれている. 労働政策研究・研修機構の調査によれば, 2020年の時点で調査対象の企業のうち技能継承を重要だと考えている企業が95\%に達している. しかしながら, 技能者の人材育成や能力開発の取り組みがうまくいっていると認識している企業は約55\%にとどまっている\cite{JILPT2020}. このような状況から, 技能伝承を実現するための取り組みが求められている. ここで技能について考えてみると, ある目的を達成するためには, それをどのように行うかを理解していること(わかる)と, 実際に行えること(できる)の両方が必要となる. このことから本研究では, 技能を, 理解と, それを実行に移すための実践的能力から構成される実践的な能力として定義する. なお, 実践的能力には身体的な動作制御から認知的な処理能力まで含まれる. このような技能を次世代に伝えていく営みが技能伝承である.\\

- 技能伝承の全体像 -\\
 技能伝承を実現するためには, 熟練技能者の持つ理解を明確化し, それを体系的に整理した上で, 育成対象の人材が実践を通じて習得し, さらにその経験を通じて新たな気づきを得るという循環的なプロセスが重要である. 本研究では, このような技能の理解に関わる部分を「知識」と表現する. これには明示的に説明可能な作業手順から, 経験に基づく暗黙的な判断基準まで含まれる.\\

- 自律的な知識抽出の必要性 -\\
 この循環的なプロセスの中でも, 熟練技能者の持つ知識を明確化し体系的に整理する段階は, その後の技能伝承活動の基盤となる重要な要素である. この知識抽出の方法について, Tatianaらは研究者と熟練技能者の関係性に基づいて, 研究者主導型, 協働型, 熟練技能者主導型の3つに分類している\cite{Tatiana2012}. このうち熟練技能者主導型(Expert-leading)は, 技能伝承を必要とする現場に比べて研究者の数が限られている現状を考慮すると, 持続可能性の観点で優位性を持つと考えられる.\\

- 知識発現 -\\
 「Expert-leading」に分類される知識抽出手法として, 西村らが提案した"知識発現"\cite{Nishimura2017}がある. この手法ではまず, 既存のマニュアルから手順を構造的に記述したプロセス知識を作成する. 次に, 作成したプロセス知識をベースとして現場の従業員が具体的な事例の紐付けと議論を行うことで, プロセス知識を改良しながら知識を抽出していく. この手法は現場が自律的に知識を抽出できる一方で, 未経験者の理解, 知識記述の限界, 記述漏れ, ワークショップ形式に起因する時間的制約, 情報システム化に向けた課題など, いくつかの実践的な課題が存在する.\\

- 提案手法の全体像 -\\
 本研究の目的は、技能伝承の現場において、熟練技能者の持つ知識を継続的に抽出・共有できる持続可能な仕組みを確立することである. 本研究では,西村らの研究\cite{Nishimura2017}で用いられたプロセス知識をベースにしながら, 技能伝承の現場における指導者と学習者の対話ログ,多量なオープンデータから構成された大規模言語モデル(LLM)を組み合わせることで,より実践的な知識の抽出手法を提案する. 具体的には技能の指導現場に着目する. まず, 技能継承の対象となる動作のベースとなる知識モデル(プロセス知識)を作成しておく. 次に, 指導現場における, 熟練技能者, すなわち指導者と学習者のやりとりを収集する. 次に指導現場で収集した情報をLLMが解析し既存のプロセス知識の改良点を生成し提案する. そしてLLMによる改良提案を受けた指導者が最終的にプロセス知識を改良する.\\

- 先行研究と提案手法の対応 -\\
 これらの要素の組み合わせにより, 先行研究で指摘された未経験者の理解に関する課題や知識記述の限界, 記述漏れのリスクなどに対して, 複合的なアプローチが可能となる. 例えば, 指導現場での対話収集により未経験者の視点を取り入れることができ, またLLMの活用により収集した対話から新たな知見を抽出することで, 知識記述の幅を広げることができる.\\

- 具体的な検証内容 -\\
 本研究では, 社交ダンスの指導現場を対象として提案手法の有効性を検証した. 具体的には, ダンススタジオの指導者2名と学習者13名の協力のもと, 3週間にわたってオンライン指導システムを用いた指導内容の収集を行った. また, 収集したデータをLLMに解析させ, 生成された改良点の提案について指導者から評価を得た. さらに, これらの結果を踏まえて指導者がプロセス知識を改良する過程についても検証を行った.\\

- 論文の構成 -\\
 本稿は以下のように構成される. 第2章では関連研究について述べ, 本研究の立場を明らかにする. 第3章では提案手法について, その理論的背景と具体的な実現方法を詳述する. 第4章では実験システムの構築とそのための予備検証について説明する. 第5章ではダンススタジオでの実証実験とその結果について述べる. 第6章では考察を行い, 最後に第7章でまとめと今後の展望を述べる.\\


