\chapter{関連研究}
\section{"知識"とその抽出に関する研究}
- 対象とする「知識」のスコーピング -\\
ポランニーは「我々は語れる以上のことを知っている(We know more than we can tell)」と述べ, 本質的に言語化できない知識の存在を指摘した\cite{Polanyi1966}. このような言語化できない知識は, 技能伝承において特に重要な課題となる.

野中・竹内が提唱したSECIモデルでは, 知識を暗黙知と形式知という2つの形態で捉え, それらの相互変換による知識創造のプロセスを説明している\cite{Nonaka1996}. このモデルでは, 暗黙知から形式知への変換(表出化), 形式知同士の組み合わせ(連結化), 形式知の実践を通じた暗黙知化(内面化), そして経験の共有による暗黙知の伝達(共同化)という4つのフェーズを通じて知識が発展していくとされる.

本研究では, 技能伝承を持続的に実現可能なものとするため, まず形式知化が可能な領域において知識抽出の手法を確立することを目指す. 本質的に言語化できない知識は, 定量的なデータ計測による分析や, 直接的な経験の共有による伝達が必要となるが, 形式知化が可能な知識は, 言語や図表による表現を通じて時間や場所の制約を超えた伝達が可能となる. 具体的には, 熟練技能者の持つ暗黙知を形式知として明示化し(表出化), それを体系的に整理する(連結化)プロセスに着目する. これらのプロセスは, その後の技能習得や新たな知識創造の土台となる.\\

- 抽出手法の分類 -\\
Tatianaらは, SECIモデルのような知識の循環的プロセスにおける「知識を明確化し, 体系的に整理する」段階に着目し, 知識抽出手法を研究者と熟練技能者の関係性に基づいて3つに分類している\cite{Tatiana2012}. 「Analyst-leading」は, 研究者が主体となって熟練技能者から知識を引き出す手法である. 具体的には, 研究者が熟練技能者にインタビューやアンケートを行うことで知識を抽出する. この手法は熟練技能者の明示的な知識を効率よく引き出せる一方で, 抽出できる知識が表層的であったり, 効果的なインタビューのためには長年の訓練と経験が必要であるといった課題がある. 「Expert-Analyst Collaborating」は, 研究者と熟練技能者が協働して知識を抽出する手法である. 具体的には, ロールプレイングゲームやバーバルプロトコルなどを通じて両者が体験を共有しながら知識を抽出する. この手法は熟練技能者が持つ本質的な知識を抽出できる一方で, 体験共有の場の準備や, 研究者が現場の専門知識を理解する必要があるなど, 多大な労力を必要とする. 「Expert-leading」は, 熟練技能者が主体となって自ら知識を抽出する手法である. 具体的には, ストーリーテリングやブレインストーミングなどを通じて知識を抽出する. この手法は研究者の介入コストが低い一方で, 抽出された知識が体系的にまとまりづらく, また熟練技能者自身が知識の言語化に苦労するという課題がある.\\

- 知識発現 -\\
このうちExpert-leadingに分類される代表的な手法として, 第1章でも述べた, 西村らの"知識発現"\cite{Nishimura2017}がある. この手法では, プロセス知識をベースとして現場の従業員が具体的な事例の紐付けと議論を行うことで, 知識を抽出していく. しかし, 以下のような実践的な課題が明らかになっている.
\begin{enumerate}
    \item 未経験者の理解に関する課題 :\\
    構造化されたプロセス知識の背景知識を持たない場合, その内容を読み解くことが困難である可能性がアンケート結果から示唆されている.
    
    \item 記述できる知識の限界に関する課題 :\\
    ワークショップ参加者間の議論のみで知識を抽出するため, 参加者が意識的に想起できる知識の総量以上のものは記述できず, 熟練技能者が無意識のうちに実践している暗黙的な知識を十分に引き出せない可能性がアンケート結果から示唆されている.
    
    \item 記述漏れに関する課題 :\\
    系統立てた記述により自然言語の自由記述に比べて記述漏れに気づきやすいものの, 完全な解決には至っていない.
    
    \item ワークショップ形式に起因する課題 :\\
    ワークショップという限られた時間と場所での議論では, 実際の業務で直面する様々な状況を網羅的に想起することが難しく, また参加者の時間的負担も大きいため, 継続的な知識抽出活動の実施が困難である.
    
    \item 情報システム化に向けた課題 :\\
    知識抽出活動を効率的に実施し継続的に改善していくためには情報システムによる支援が必要であるが, 事例とプロセス知識との関連性を直感的に表現できるUIの実現や, より効果的な知識想起を促す知識モデルの設計など, 実用化に向けた技術的な課題が残されている.
\end{enumerate}

これらの課題に対し, 本研究では指導現場に着目することとLLMを活用することで複合的な解決アプローチを提案する. 

まず, 未経験者の理解に関する課題に対しては, 指導現場での学習者からの質問や反応を通じて, 未経験者が理解しづらい点を直接的に把握できる. またLLMが収集した対話から理解を妨げる要因を分析し, より理解しやすい形での知識の表現を提案することができる.

知識記述の限界に対しては, 指導現場での予期せぬ質問が熟練技能者の無意識の知識を引き出すきっかけとなり, さらにLLMが収集した対話から新たな知見を抽出することで, 知識記述の幅を広げることができる. 

記述漏れの課題に対しては, LLMが収集された対話を分析することで, 体系的な観点から漏れている知識を指摘できる可能性がある. また, 指導現場という実践の場での継続的な対話収集により, 様々な状況における知識を網羅的に収集できる.

ワークショップ形式に起因する課題に対しては, 指導現場という日常的な文脈で知識抽出を行うことで, 時間的制約や想起の困難さを軽減できる. 

最後に情報システム化に向けた課題に対しては, LLMを活用することで, 収集した対話から知識モデルの改良点を自動的に提案できる可能性がある.\\

- その場その場での記録 -\\
知識抽出に関する他の研究として, 内平ら\cite{Uchihira2022}は, 日常的なつぶやきを音声情報として収集し知識を獲得する手法を提案している. この手法の特徴は, 業務中の気づきや閃きをその場で自然に収集できる点にあり, 知識抽出をより日常的な活動として実現している.  

本研究も同様に, 指導現場という自然な文脈での即時的な知識抽出を目指すが, さらにLLMを活用することで収集した知識の分析と体系化を行う点に新規性がある. また, 内平らの提案する手法と本研究は, どちらも日常的な活動の中で自然に知識を収集するという点で共通しており, これらを組み合わせることでより包括的な知識抽出システムを実現できる可能性がある. 例えば, 指導現場での気づきを音声情報として収集しながら, 指導内容をLLMで分析・体系化するといった相補的な活用が考えられる.\\



\section{指導現場に着目した研究}
これまで指導現場で生じるインタラクションに関する研究も行われてきた.Moore\cite{Moore1989}は, 教育における相互作用を学習者-コンテンツ間, 学習者-指導者間, 学習者間の3つに分類し, 特に学習者-指導者間の相互作用において, 指導者は自身の経験に基づいて学習者の理解を促進させることを指摘している.この知見は, 指導現場における相互作用が, 熟練者の持つ知識を表出させる機会となる可能性を示唆している.

広瀬・深澤\cite{Hirose2018}は, 指導における言語使用について, 科学言語(直示的な表現)とわざ言語(比喩的な表現)という観点から理論的な分析を行っている.特に, 指導場面における対話的状況では, 発話者が想定する「現実」を学習者と共有することの重要性を指摘している.ただし, この研究は理論的な枠組みの提示に留まっており, 実際の指導現場での言語使用を通じた知識抽出については検討されていない.

佐伯ら\cite{Saeki2017}は, 指導者の成長プロセスに着目し, 経験と省察を通じて指導力が向上していく過程を分析している.この研究は指導者の経験が言語化される過程を示唆しているが, 指導現場でのインタラクションを通じた知識抽出という観点からの分析は行われていない.

これらの先行研究は, 指導者の知識や経験に関する重要な知見を提供している一方で, 実際の指導現場でのインタラクションを通じて熟練者の知識を抽出する方法については十分な検討がなされていない.特に, 学習者との相互作用の中で熟練者がどのように自身の知識を表出させ, それをどのように抽出できるのかについては, さらなる検討が必要である.
本研究では, これまでの研究で十分に検討されてこなかった指導現場でのインタラクションに着目し, そこから熟練者の持つ知識を抽出する方法を明らかにすることを目指す.\\



\section{LLMを活用したナレッジマネジメントシステムの研究}
近年, LLMの発展により様々な知的活動への活用が研究されている. 大きく分けると,(1)創造性支援ツール(CST)としての活用,(2)LLMの性能向上に関する研究, (3)知識ベース構築への活用という3つの方向性がある.

創造性支援ツール(CST)としての活用について, 田中らはマルチエージェントのアプローチを提案している\cite{Tanaka2024}. この手法では, 異なる役割を持つLLMエージェントが対話的にアイデアを生成することで, 単一のエージェントでは得られない多様な発想を引き出すことができる. また, 尾崎らはLLMに反論文を生成させることで, 議論における批判的思考を補完する手法を提案している\cite{Ozaki2024}. これらの研究から, LLMを適切に活用することで発散的思考や収束的思考の支援, さらには認知バイアスの軽減などの効果が期待できることが示唆されている.

LLMの性能向上に関して, WelzらやLewisらは外部知識ベースを活用することで, LLMが生成するコンテンツの事実性を向上させる手法を提案している\cite{Welz2024, Lewis2020}. これらの研究は, LLMと既存の構造化された知識を組み合わせることの有効性を示している.

知識ベース構築への活用においては, ShenとLin\cite{Shen2024}は, LLMを搭載した個人用知識アシスタント(K-assistant)を提案している. このシステムでは, 自然言語処理, 理解, 推論の機能を組み合わせることで, 実践コミュニティ(CoP)のメンバー間の知識活動を支援する.特に, 暗黙知から形式知への変換プロセスを促進する点で, 従来の知識抽出手法とは異なるアプローチを取っている.また, 鵜飼らはLLMを使用して知識ベースそのものを構築する手法を提案している\cite{Ukai2023}が, 現状のLLMの性能では知識ベースをゼロから生成することは困難であり, 人間の専門知識による検証や修正が必要とされている.

これらの先行研究に対し, 本研究の新規性は以下の点にある.第一に, 従来研究の多くがLLMの性能向上や汎用的な知識ベースの構築を目指していたのに対し, 本研究は熟練技能者特有の知識抽出に焦点を当てている.第二に, LLMを知識抽出の補助ツールとして位置づけ, 人間とLLMが相互に補完し合いながら知識を抽出・構造化するプロセスを提案している. 特に、指導現場での対話分析を通じて、熟練者が持つまだ言語化されていない知識を、LLMを活用して体系的に抽出・整理する点は、これまでにない新しいアプローチである。