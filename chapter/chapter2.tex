\chapter{関連研究}
\section{知識創造と抽出に関する研究}
- 対象とする「知識」のスコーピング -\\
 ポランニーは「我々は語れる以上のことを知っている(We know more than we can tell)」と述べ, 本質的に言語化できない知識の存在を指摘した\cite{Polanyi1966}. このような言語化できない知識は, 技能伝承において特に重要な課題となる. \\
 野中・竹内が提唱したSECIモデルでは, 知識を暗黙知と形式知という2つの形態で捉え, それらの相互変換による知識創造のプロセスを説明している\cite{Nonaka1996}. このモデルでは, 暗黙知から形式知への変換(表出化), 形式知同士の組み合わせ(連結化), 形式知の実践を通じた暗黙知化(内面化), そして経験の共有による暗黙知の伝達(共同化)という4つのフェーズを通じて知識が発展していくとされる.\\
 本研究では, 技能伝承を持続的に実現可能なものとするため, まず形式知化が可能な領域において知識抽出の手法を確立することを目指す. 本質的に言語化できない知識は, 定量的なデータ計測による分析や, 直接的な経験の共有による伝達が必要となるが, 形式知化が可能な知識は, 言語や図表による表現を通じて時間や場所の制約を超えた伝達が可能となる. 具体的には, 熟練技能者の持つ暗黙知を形式知として明示化し(表出化), それを体系的に整理する(連結化)プロセスに着目する. これらのプロセスは, その後の技能習得や新たな知識創造の土台となる.\\

- 抽出手法の分類 -\\
 Tatianaらは, SECIモデルのような知識の循環的プロセスにおける「知識を明確化し, 体系的に整理する」段階に着目し, 知識抽出手法を研究者と熟練技能者の関係性に基づいて3つに分類している\cite{Tatiana2012}. 「Analyst-leading」は, 研究者が主体となって熟練技能者から知識を引き出す手法である. 具体的には, 研究者が熟練技能者にインタビューやアンケートを行うことで知識を抽出する. この手法は熟練技能者の明示的な知識を効率よく引き出せる一方で, 抽出できる知識が表層的であったり, 効果的なインタビューのためには長年の訓練と経験が必要であるといった課題がある. 「Expert-Analyst Collaborating」は, 研究者と熟練技能者が協働して知識を抽出する手法である. 具体的には, ロールプレイングゲームやバーバルプロトコルなどを通じて両者が体験を共有しながら知識を抽出する. この手法は熟練技能者が持つ本質的な知識を抽出できる一方で, 体験共有の場の準備や, 研究者が現場の専門知識を理解する必要があるなど, 多大な労力を必要とする. 「Expert-leading」は, 熟練技能者が主体となって自ら知識を抽出する手法である. 具体的には, ストーリーテリングやブレインストーミングなどを通じて知識を抽出する. この手法は研究者の介入コストが低い一方で, 抽出された知識が体系的にまとまりづらく, また熟練技能者自身が知識の言語化に苦労するという課題がある.\\

- 抽出の場の観点での分類 -\\
また, 知識を抽出するための場という観点に着目すると, ワークショップ形式の手法がある\cite{Nishimura2017,Yoshida2022,Ijuin2022,Uchihira2022}. この手法では基本的に, まず, 普段の業務を動画や音声等で記録する. 次に定期的に開催されるワークショップの中で, 普段から記録していた動画や音声を基に当時の状況を想起し, 知識を体系化する必要がある. しかし, 現場の意思決定プロセスは非常に複雑であり, 経験に基づいたパターン認識や直感的判断を用いて迅速に状況を評価し行動を選択している\cite{Klein2008}. こうした当時の複雑な意思決定を想起させるための十分な情報が動画や音声に含まれているとは限らない. そのため, ワークショップ形式には動画や音声に記録されない情報を忘却してしまった場合, 当時の状況を想起することが難しくなるといった課題がある.
こうした課題に対し, 本研究では指導者と学習者, そしてLLMが常に知識モデルを共有し, さらにそれらが相互作用可能なシステムを構築するため, その場で知識発現が可能な点で新規性がある.

\section{指導現場で表出する知識}

\section{知識抽出におけるLLMの活用}
 また近年, LLMを創造性支援ツール(CST)として活用するための研究が数多く行われている. 例えば, アイディエーション支援に使用している例がある\cite{Tanaka2024}.こうした研究ではマルチエージェントのLLMに多様なアイデアを提案させ, 人の発散的思考を助ける狙いがある. また, LLMに反論文を生成させることで, 議論における批判的思考を補おうとする研究もある\cite{Ozaki2024}. これらの研究から分かるように, LLMを適切に使用することで発散/収束思考の支援や認知バイアスの軽減などの効果が期待できる. 
 一方でLLMが生成するコンテンツの事実性向上のために外部知識ベースを活用する研究がある\cite{Welz2024, Lewis2020}. より発展したものとして, LLMを使用して知識ベースそのものを構築する取り組みもある\cite{Ukai2023}が, 現在のLLMの性能では知識ベースをゼロから生成することはできていない. また, 多くの研究が汎用的な知識ベースの構築を目指している.  
 こうした研究に対し, 本研究は人間とLLMが補い合って知識ベースを構築する点, さらに熟練技能者特有の知識ベースを構築するという点で新規性がある.

\cite{Nishimura2008}