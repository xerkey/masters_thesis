\chapter{まとめと今後の展望}
 本研究の目的は, 熟練技能者の持つ知識を自然な文脈の中で抽出し, 整理する手法を確立することであった.この目的を達成するため, 本研究では「指導現場での対話収集」と「LLMによる知識抽出支援」を組み合わせた新しい手法を提案し, 社交ダンスを対象とした実証実験を通じてその有効性を検証した.\\
 提案手法の特徴は, 第一に指導現場というワークショップよりも自然に知識が表出する文脈に着目した点にある. 従来の知識発現手法では, ワークショップという特別な場を設定し, 意図的な想起と議論に基づいて知識抽出を行う必要があった. これに対し本研究では, 指導者と学習者の自然なやり取りの中から知識を抽出することで, より自然な文脈での知識抽出の実現を目指した. 第二の特徴は, LLMを知識抽出支援ツールとして活用した点である. これにより, 収集された対話から新たな知見を抽出し, それを既存の知識体系に統合していく過程を支援することを試みた. \\
 実証実験では, 石川県内のダンススタジオの協力を得て, 指導者2名と学習者13名による約5週間の実践を通じてデータを収集した. 分析の結果, 以下の点が明らかになった.\\
 第一に, 指導現場での知識抽出の実現可能性が確認された. 収集された指導コメントの76\%に未含有要素が含まれており, 普段の指導活動の中で新たな知識要素を自然に抽出できることが示された. 特に学習者からの質問が, 指導者の暗黙知を引き出すきっかけとなる場面も観察された. また, 同じ技術要素が学習者の状況に応じて多様に表現されるという, 指導現場特有の知識抽出パターンも確認された.\\
 第二に, LLMによる知識抽出支援の有効性が示された. LLMは未含有要素の抽出において高い精度(F値0.928)を示し, 知識の体系的な整理に貢献できることが確認された. ただし, 指導の本質的な意図の理解には限界があり, 人間の専門的な解釈が必要不可欠であることも明らかになった.\\
 さらに重要な発見として, 社交ダンスにおける知識構造の二重性が示唆された. すなわち, 時系列的な動作の連鎖という側面と, その背後にある原理的な知識という側面が存在する可能性がある. この発見に基づき, 技能を「目標層」「原理層」「認知層」「実行層」という4層で捉える分類枠組みと, 実行層の特性による分類枠組みを提案した. これらの枠組みは, 技能分野の特性に応じた適切な知識表現方法の選択に貢献するものである.\\
 特に社交ダンスについては, 時系列的な動作を単層的に記述し, その背後にある原理的な知識をCHARMや目的指向知識の枠組みで表現するという, 二元的なアプローチを提案した. このアプローチは, 社交ダンスに限らず, 同様の特性を持つ他の身体技能分野にも適用できる可能性がある.\\
 本研究の主な貢献は以下の3点にまとめられる.\\
\begin{enumerate}
    \item 自然な文脈での知識抽出の仕組みの提案:\\
    従来のワークショップ形式での知識抽出に代わり, 指導現場での対話とLLMを組み合わせた新しい知識抽出の枠組みを提示した. これにより, より自然な文脈での技能伝承の仕組みの実現可能性を示した.

    \item 身体技能における知識構造の解明:\\
    社交ダンスを事例として, 身体技能における知識構造の二重性を明らかにした. また, この知見に基づいて技能分野を体系的に分類する枠組みを提案し, 技能伝承研究の理論的基盤の拡充に貢献した.

    \item 知識表現方法の選択指針の提示:\\
    技能の特性に応じた適切な知識表現方法の選択指針を示した. 特に, 時系列的な動作記述と原理的な知識記述を分離するという新しいアプローチを提案し, その有効性を示した.
\end{enumerate}
 今後の課題としては, 以下の点が挙げられる.\\
\begin{enumerate}
    \item マルチモーダルなデータ収集・分析の実現:\\
    動作の映像データや非言語情報など, より多角的なデータの収集と分析を行うことで, より豊かな知識体系の構築を目指す必要がある.

    \item LLMの活用方法の高度化:\\
    専門分野に特化した事前学習や, 指導者の意図理解に焦点を当てたプロンプト設計の改善など, LLMの活用方法をさらに発展させる必要がある.

    \item 二元的アプローチの検証と展開:\\
    提案した時系列的動作記述と原理的知識記述の二元的アプローチについて, 実践的な検証を行うとともに, 他の技能分野への展開可能性を探る必要がある.
\end{enumerate}
 これらの課題に取り組むことで, より効果的な技能伝承の実現に向けて研究を発展させる可能性がある.\\
