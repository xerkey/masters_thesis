\chapter{考察}
\section{提案手法の本質的な特性と課題}
 本研究で提案した手法は, 指導現場での知識抽出とLLMによる支援を組み合わせた点に特徴がある. 本節では, この手法における知識抽出のメカニズムとLLMの役割について, その本質的な特性を考察する.

\subsection{指導現場における知識抽出の特性}
 指導現場における知識抽出のプロセスは, 定型的なワークショップとは異なる特徴を持つことが明らかになった. 特に注目すべき点は, 学習者からの質問が新たな知識要素の表出のきっかけとなっている点である. 例えば今回の実験では, 「体重移動しながら回転させてしまっている」という学習者からの質問に対し, 指導者が「歩幅を少し狭めた方が体重移動とヒップの回転を分けられる」という, プロセス知識には含まれていなかった重要な技術要素を提示するという場面が観察された. このような対話を通じた知識の表出は, 指導現場ならではの特徴といえる.\\
 また, 指導現場特有の知識抽出パターンとして, 同一の技術要素が学習者の状況に応じて多様に表現される現象が観察された. 熟練した指導者は, 学習者の動きに応じて技術要素を様々な言葉で説明していた. 例えば「つま先を外側に向ける」という技術要素は, ある場面では「左右のかかとが離れないようにつけて踏みかえ」として, また別の場面では「クローズする足のつま先が内側に向かないようによせる」として説明されていた. \\
 このような状況に応じた表現の使い分けは, 広瀬・深澤\cite{Hirose2018}が指摘する指導場面における「科学言語」と「わざ言語」の使い分けと類似している. 彼らは, 指導場面における対話的状況では, 発話者が想定する「現実」を学習者と共有することの重要性を指摘している. また佐藤ら\cite{Sato2024}は, インタビューによって得られた理想動作の知識構造化に加えて, 実際の指導場面での新たな知識の獲得可能性を示している. 本研究で観察された現象は, これらの知見を支持するとともに, 指導現場での対話を通じて具体的にどのような知識が抽出されうるかを示すものといえる.\\
 さらに, このような表現の使い分けを通じて, 技術要素そのものの理解だけでなく, それを効果的に伝えるための指導者としての専門的知識が抽出できる可能性が示された. これは佐伯ら\cite{Saeki2017}が分析した指導者の成長プロセスにおける「経験と省察を通じた指導力の向上」とも整合する知見である. つまり, 指導現場での対話を通じて抽出される知識には, 技術的な要素に加えて, その技術をいかに効果的に伝えるかという指導方略に関する暗黙知も含まれているといえる.\\
 このように, 技術要素が具体的な指導場面で多様に表現され, それらが蓄積されることで, 技術の本質的な理解と効果的な指導方法の両面において, より包括的な知識の獲得が可能となる. これは従来の技能伝承研究では十分に注目されてこなかった, 指導者固有の知識という新たな側面を浮き彫りにするものである.\\
 また, 提案手法における知識表出の実践的な効果と限界も明らかになった. 効果としては, 指導者と学習者の双方にとって普段のレッスンの延長線上で知識を表出できること, レッスン時間外でも随時コメントの追加や質問が可能であること, そして動画を見ながら具体的な動作の瞬間を指摘できることが挙げられる. その一方で, オンライン環境では実際の動作を直接修正できないこと, スマートフォンでの動画再生とコメント入力の切り替えが煩雑であること, といった技術的な制約も確認された. 特にオンライン環境においては, 対面指導のような即座の質疑応答や直接的な動作修正が困難なため, 質疑の活発さが低下する傾向が見られた. これは, 技能指導における直接的な相互作用の重要性を示唆している.\\
 今後の展望として, 指導現場での知識抽出の可能性をより広げるために, マルチモーダルなデータ収集方法の導入が考えられる. 本研究では指導者と学習者の対話から知識を抽出したが, 実際の指導場面では言語的な説明以外にも多くの情報が存在する. 例えば, 動作の映像データと指導者の説明を同期させた分析により, 「つま先を外側に向ける」といった言語化された指示が実際の動作のどの局面でどのように実現されているかを定量的に把握することが可能となる. また, 指導者のジェスチャーや視線の動きなどの非言語情報を分析することで, 言語化されにくい知識の一端を捉えられる可能性がある.

\subsection{LLMを活用した知識抽出支援の現状と展望}
 本研究におけるLLMは, 人間の知識抽出活動を補完しつつも, 完全な自動化を目指すものではないという特徴を持つ. 実験結果から, LLMは指導現場で得られた情報から未含有要素を高い精度で抽出できることが確認された一方で, 指導の背景にある本質的な意図の理解には限界があることも明らかになった.\\
 このことは, 人間とLLMの相補的な関係性を示唆している. LLMは大量の指導事例を効率的に処理し, 表層的な知識要素を抽出することで, 人間の分析を支援する. 特に, 指導事例の蓄積が進むにつれて, LLMによる網羅的な分析の価値は相対的に高まることが期待される. 一方, 指導の本質的な意図の理解や, 技術要素間の関係性の把握は人間が担う. このような役割分担により, より効果的な知識抽出が実現可能となる.\\
 現状のLLMが示す限界は, その学習方式に起因すると考えられる. LLMは大規模なテキストデータから統計的なパターンを学習することで言語理解を実現しているが, 技能指導の文脈では同じ意図を持つ表現であっても, その場の状況(学習者の動き方, 習熟度, 理解度など)に応じて大きく異なる表現が用いられる. こうした状況依存的な表現の変化は, 一般的な言語データでは出現頻度が低く, 統計的な関連性として捉えることが困難である. しかし, この課題に対しては技術的な改善の可能性が残されている. 例えば, プロンプトの最適化により状況の文脈をより明確に考慮したり, 外部知識ベースとの連携(RAG)により専門分野特有の表現パターンを補完したり, 特定分野向けのファインチューニングにより技能指導特有の表現変化を学習したりすることで, より深い知識抽出支援が実現される可能性がある.\\
 さらに, 本研究ではLLMからの一方向的な改良提案という形式を取ったが, より効果的な知識抽出支援の実現に向けては, 指導者とLLMの双方向的なコミュニケーションが有効である可能性がある. 例えば, LLMが提案した改良点について指導者が疑問や反論を示し, それに対してLLMがより詳しい説明や代替案を提示するという対話的なプロセスを通じて, より本質的な技術要素の抽出や, 指導者自身も意識していなかった知識の表出を促すことができると考えられる. このような対話的アプローチは, 田中ら\cite{Tanaka2024}が提案する異なる役割を持つLLMエージェント間の対話による創造的な知識生成の考え方とも整合する. 彼らは, 複数のエージェントが対話的にアイデアを生成することで, 単一のエージェントでは得られない多様な発想を引き出せることを示している. 同様に, 指導者とLLMの対話的な知識抽出プロセスは, 単なる知識の抽出や整理を超えて, 新たな気づきや省察を促す可能性を秘めている.\\



\section{技能伝承における知識構造の再考}
\subsection{CHARMの適用における本質的な課題}
 本研究の実験結果は, 社交ダンスにおける指導現場でのインタラクションを通じて新たな知識を抽出できることを示した一方で, CHARMによる知識表現に一定の限界があることも示唆した. この限界は単なる手法の問題ではなく, 技能伝承における知識構造の本質に関わる重要な示唆を提供している. \\
 CHARMは本来, 行為間の目的達成関係を記述するためのモデルであり, 「ある目的を達成するために必要な行為の系列」を階層的に表現することを意図している. 例えば介護現場での食事介助では, 「安全に食事を提供する」という目的に対して, 「誤嚥を防ぐ」という中間的な目的があり, それを達成するために「適切な姿勢を保持する」「一回の食事量を調整する」といった具体的な行為が必要となる. これらの行為は時間的な順序ではなく, 目的を達成するための論理的な必然性によって結びついている. 同様に, 製造業では「品質の高い製品を生産する」という目的に対して, 「精度の高い加工を実現する」という中間的な目的があり, それを達成するために「工具の状態を確認する」「適切な加工条件を設定する」といった行為が必要となる. このように, 目的と手段の論理的な階層関係を記述することがCHARMの本来の使用法である. \\
 しかし, 本研究で作成した社交ダンスのプロセス知識は, 結果的にCHARMの本来の意図から外れた使用方法となっていた. 例えば, クカラチャの動作において「右足に立つ」「左足を左へ送る」「左足を横にステップする」という行為の連鎖は, CHARMの階層構造として記述されている. しかし, これらの行為は上位の目的を達成するための手段という関係にはない. むしろ, これらは時間軸に沿って順序付けられた一連の動作の分解に過ぎない. さらに重要な点として, 「軸に関して:背骨が上から吊り下がっている感じ, 頭と背骨が床に対して垂直になるイメージ」といった技能全体を通じて維持すべき重要な原理的要素が, 階層構造の中で適切に表現できず, 最上位のノードの属性として記述せざるを得なかった. \\
 このような問題の分析を通じて, 社交ダンスのような身体技能の分野では, 知識構造に異なる性質を持つ二つの側面が存在することが明らかになった. 一つは時系列的な動作の連鎖である. 例えば「クカラチャ」という動作は, 特定の身体の動きが, 決められた時間的順序で実行されることで初めて成立する. この時系列的な順序は, 目的達成のための論理的な必然性というよりも, その技能に固有の約束事として存在している. \\
 もう一つは原理的な知識の階層構造である. 本研究で明らかになった「つま先を外側に向ける」「軸を床に対して垂直に保つ」「脚をだす, 体重を移動する, ヒップを回転するを分ける」「常に床に力をかけ続ける」という四つの技術要素は, 個々の動作の時系列的な順序とは独立に, その技能全体を支える原理として機能している. これらの要素は特定の時点での動作としてではなく, 継続的に維持すべき要件として存在する. \\
 この分析は, 技能の知識構造をより適切に表現するための方向性を示唆している. すなわち, 原理的な要素(「なぜその動作が必要か」「どのような効果を意図しているか」)の記述にはCHARMの目的-手段の階層構造を活用し, 一方で具体的な動作の時系列的な記述には別の表現形式を用いるという, 二元的なアプローチが必要である. \\
 このような知識構造の二重性を踏まえ, より体系的な理解のための枠組みが必要となる. 以下では, まず技能分野全般を分類するための階層モデルを提示し, 次に実際の動作がどのような特性を持つかという観点からの分類を行う. これらの分類枠組みは, 各技能分野に適した知識表現方法を選択するための指針となる. 社交ダンスの場合, 時系列的な動作の記述には単層的な表現を用い, その背後にある原理的な要素の記述にはCHARMの目的指向的な記述を組み合わせるという方法が適切であると考えられる. \\


\subsection{技能分野における知識の階層性}
 まず, 技能を4つの階層で捉える知識構造モデルを提案する. 具体的には以下の4層である.
\begin{enumerate}
\item 目標層 : その分野が最終的に目指すもの, 評価基準となるもの
\item 原理層 : 目標を実現するための基本的な法則や原理
\item 認知層 : 人が持つ内的な判断基準やメンタルモデル
\item 実行層 : 実際に観測可能な行動や動作
\end{enumerate}
 この4層構造に基づき, 各層の知識が形式化されているか(明示的か), されていないか(暗黙的か)によって技能分野を分類することができる. 表\ref{table_skill_hierarchy}は, その組み合わせによる16パターンを示したものである. ここで○は明示的(形式化されている), ×は暗黙的(形式化されていない)を表す.\\
 例えば単純な機械操作では, すべての層が形式化されており(○,○,○,○), 完全な自動化が可能である. 一方, 本研究で対象とした社交ダンスのような分野では, 実行層の動作は観察可能である(○)ものの, 他の層は形式化されていない(×,×,×). すなわち目標層における評価基準, 原理層における動作の原理, 認知層における指導者の判断基準が, いずれも暗黙知, もしくは形式化されていない知識として存在している.\\
 製造業の技能伝承では, 目標層は製品の機能や性能として明確に定められており(○), 原理層も物理法則として解明されている場合が多い(○). しかし認知層, すなわち熟練技能者がどのように状況を判断し対応を決定しているかという内的なプロセスは暗黙知として残されている(×)場合が多い. また実行層である具体的な作業動作は観察可能である(○).\\
 このように技能を階層的に捉え, 各層の形式化状況を分析することで, 技能伝承における課題をより具体的に特定することができる. さらに, この分類は知識抽出のアプローチ方法の選択にも示唆を与える. すなわち, 上位層が形式化されている分野では目標からの演繹的なアプローチ(トップダウン型)が可能である一方, 上位層が暗黙的な分野では実行層からの帰納的なアプローチ(ボトムアップ型)が必要となる.\\


\subsection{実行層の特性による分類}
 前項で示した形式化状況による分類に加えて, 実行層の基本的特性による分類も重要である. 特に「連続性」と「順序性」という2つの観点から技能分野の特徴を捉えることで, より適切な知識表現方法の選択が可能となる. \\
 連続性とは, 実行要素を明確に区切れるか否かを示す. 例えば料理では「材料を切る」「調味料を加える」のように, 各作業を明確に区切ることができる. 一方, 社交ダンスでは動作が連続的に流れており, 個々の動作の境界を明確に区切ることは困難である. \\
 順序性とは, 要素間に明確な順序関係があるか否かを示す. プログラミングのように論理的な依存関係から順序が強く制約される分野がある一方で, 絵画制作のように作業の順序が比較的自由な分野も存在する. \\
 これら2つの軸により, 技能分野は表\ref{table_process_classification}に示すように4つに分類できる. 「連続的・順序性強」には社交ダンスやバイオリン演奏が該当し, 行為の境界が明確でなく切り分けが困難でありながら, 一定の順序関係が存在する. 「連続的・順序性弱」には格闘技や理学療法が該当し, 行為は連続的でありながら状況に応じて順序が変化する. 「離散的・順序性強」にはプログラミングや料理が該当し, 明確なステップと定められた順序関係を持つ. 「離散的・順序性弱」には華道や看護が該当し, 個々の作業は明確だが状況に応じて順序を選択可能である. \\
 この分類から, 社交ダンスのような「連続的・順序性強」の分野に対してCHARMを適用する際の本質的な課題が明らかになる. CHARMは本来, 離散的な作業ステップを階層的に表現することを意図して設計されている. そのため, 連続的な動作を離散的なステップに分解して表現せざるを得ず, 動作の連続性や流れを十分に表現できない. このことは本研究で明らかになった4つの重要な技術要素(「つま先を外側に向ける」「軸を床に対して垂直に保つ」「脚をだす, 体重を移動する, ヒップを回転するを分ける」「常に床に力をかけ続ける」)の扱いにも表れている. これらの要素は動作全体を通じて継続的に意識すべき前提条件であり, 離散的なステップの一部として表現することは本質的に困難である. \\
 このような分析は, 技能分野の特性に応じた知識表現方法の選択の必要性を示唆している. 例えば離散的で順序性の強い分野では, CHARMのような階層的なプロセス表現が効果的である. 一方, 連続的な要素を含む分野では, 動作の連続性や前提条件を適切に表現できる新たな知識表現方法が必要となる. \\

\begin{table}[htbp]
    \setlength{\tabcolsep}{4pt}
    \small
    \renewcommand{\arraystretch}{1.4}
    \centering
    \begin{tabular}{c|c|c|c|p{3.5cm}|p{4.5cm}}
        \hline
        目標層 & 原理層 & 認知層 & 実行層 & 該当分野の例 & 特徴 \\ \hline\hline
        ○ & ○ & ○ & ○ & 単純な機械操作 & 全ての層が解明・形式化され, 完全な自動化が可能 \\ \hline
        ○ & ○ & ○ & × & 産業用ロボットの教示作業 & 何をすべきかは明確だが, 実行時の微調整が必要 \\ \hline
        ○ & ○ & × & ○ & スポーツの基本動作 & 理想的な動きは知られているが, 習得プロセスは個人差大 \\ \hline
        ○ & × & ○ & ○ & 熟練工の検品作業 & 目標は明確で動作も観察可能だが, 判断基準が暗黙的 \\ \hline
        × & ○ & ○ & ○ & 現代アート制作 & 手法は確立されているが, 何を表現するかは作家次第 \\ \hline
        ○ & ○ & × & × & 高度な機械加工 & 目標と原理は明確だが, 実現方法に熟練を要する \\ \hline
        ○ & × & ○ & × & 接客サービス & 目標は明確だが, その実現方法は状況依存 \\ \hline
        ○ & × & × & ○ & スポーツの戦術執行 & 勝利という目標は明確だが, 実現プロセスは複雑 \\ \hline
        × & ○ & ○ & × & 即興音楽演奏 & 音楽理論は明確だが, 表現と実行は個人的 \\ \hline
        × & ○ & × & ○ & 伝統工芸の基礎技法 & 基本的な技法は確立されているが, その価値は解釈次第 \\ \hline
        × & × & ○ & ○ & ストリートダンス & 個人の表現は明確だが, 評価基準は主観的 \\ \hline
        ○ & × & × & × & 研究開発職 & 目標は明確だが, 実現方法は試行錯誤 \\ \hline
        × & ○ & × & × & 芸術作品の模写 & 技法は明確だが, 価値や習得は個人的 \\ \hline
        × & × & ○ & × & カウンセリング & 対人的な判断は可能だが, 他は状況依存 \\ \hline
        × & × & × & ○ & 創作ダンス & 動きは観察可能だが, 他の要素は主観的 \\ \hline
        × & × & × & × & 新しい芸術分野の開拓 & 全ての層が未解明で探索的\\
        \hline
    \end{tabular}
    \caption{技能における4層構造の分布と特徴}
    \label{table_skill_hierarchy}
\end{table}

\begin{table}[htbp]
    \setlength{\tabcolsep}{4pt}
    \small
    \renewcommand{\arraystretch}{1.5}
    \centering
    \begin{tabular}{|c|c|c|}
        \hline
        & 順序性強 & 順序性弱 \\ \hline
        連続的 & 
        \begin{tabular}{@{}l@{}}
            \textbf{特徴:}\\
             行為の境界が明確でなく切り分けが困難,\\
             かつ順序関係が重要\\
            \textbf{具体例:}\\
             ・社交ダンス※\\
             ・バイオリン演奏※
        \end{tabular} &
        \begin{tabular}{@{}l@{}}
            \textbf{特徴:}\\
             行為の境界が明確でなく切り分けが困難,\\
             かつ状況に応じて順序が変化\\
            \textbf{具体例:}\\
             ・格闘技※\\
             ・理学療法※
        \end{tabular} \\ \hline
        離散的 & 
        \begin{tabular}{@{}l@{}}
            \textbf{特徴:}\\
             明確なステップと定められた順序関係を持つ\\
            \textbf{具体例:}\\
             ・料理\\
             ・プログラミング
        \end{tabular} &
        \begin{tabular}{@{}l@{}}
            \textbf{特徴:}\\
             個別の作業は明確で状況に応じて\\
             順序を選択可能\\
            \textbf{具体例:}\\
             ・華道\\
             ・看護
        \end{tabular} \\ \hline
    \end{tabular}
    \caption{実行層の特性による技能分野の分類(※は身体動作を伴うもの)}
    \label{table_process_classification}
\end{table}

\subsection{社交ダンスにおける知識表現方法の提案}
 これまでの分析により, 技能の特性に応じた適切な知識表現方法の選択が重要であることが明らかになった. 特に社交ダンスのような身体技能においては, その知識構造の二重性を適切に表現できる方法が必要となる.\\
 本研究で示した二つの分類枠組みから, 社交ダンスは以下のような特徴を持つことが分かる. まず4層構造における形式化状況の観点では, 実行層のみが形式化され(○), 目標層, 原理層, 認知層は暗黙的(×)である. これは, 動作そのものは観察可能であるものの, その背後にある原理や目標が十分に形式化されていない状態を示している. また実行層の特性という観点では, 「連続的・順序性強」に分類され, 動作が連続的に流れる一方で一定の順序関係を持つという特徴を示している.\\
 これらの特徴を踏まえ, 社交ダンスにおける知識表現は以下の二つの側面を明確に分離して扱うべきである. 第一に時系列的な動作の記述である. 動作を単層的な時系列構造として記述することで, CHARMで誤って表現していた目的-手段の階層関係から解放され, 動作の本質的な連続性を適切に表現することが可能となる. 具体的な表現形式としては, まず基本となる動作の系列を言語情報として記述し, そこにモーションキャプチャなどの定量的な動作データをセグメンテーションによって紐付けていく方法が考えられる. これにより, 動作の連続性を保持しながら, 必要に応じて離散的な分析も可能となる.\\
 第二に原理的な要素の記述である. これには伊集院ら\cite{Ijuin2022}の研究で提案された目的指向知識の枠組みを拡張して活用する. 従来のCHARMでは一つの行為に一つの目的しか紐付けられなかったが, 目的指向知識では複数の目的を持つことができる. これにより, 例えば「つま先を外側に向ける」という動作が「バランスを保つ」「美しく見せる」「次の動作への準備をする」といった複数の目的を持つことを表現できる. このような多目的性の表現は, これまで暗黙的になっていた認知層(熟練者の判断基準)や原理層(動作の原理)の形式化につながる可能性がある. さらに, それらの目的間の関係性を分析することで, 社交ダンスという技能の目標層における価値基準の解明にも寄与すると考えられる.\\
 このような二元的なアプローチは, 技能の段階的な習得においても有効性を発揮すると考えられる. 学習者はまず時系列的な動作の系列を基本的な枠組みとして習得し, その後, 各動作が持つ複数の目的を理解していくことで, より深い技能の理解と習得が可能となる. 特に, ある動作が複数の目的を持つことの理解は, 状況に応じた適切な動作の選択や, 新しい状況への対応力の向上につながると期待される.\\
 今後の課題として, まずはこの二元的なアプローチの有効性を実践的に検証する必要がある. 具体的には, 実際の指導現場でこの枠組みを用いた知識記述を試み, その効果を評価することが重要である. その上で, 時系列的な動作記述と目的指向的な知識記述を統合的に扱えるシステムの開発へと進むことが望ましい. このアプローチは社交ダンスに限らず, 本研究で示した分類枠組みにおいて類似の特徴を持つ技能分野への適用が期待できる. 例えば, バイオリン演奏のような「連続的・順序性強」の分野や, 実行層のみが形式化された他の身体技能において, 同様の知識表現方法が有効である可能性がある. さらに, 各技能分野の特性に応じて本アプローチを適切に調整することで, より広範な技能伝承の課題解決に貢献できると考えられる.\\

% \section{技能分野の特性に基づく知識表現方法の選択}
%  本研究の実験結果は, 社交ダンスにおける指導現場でのインタラクションを通じて新たな知識を抽出できることを示した一方で, CHARMによる知識表現に一定の限界があることも示唆した. 特に指導事例の分析から, 「つま先を外側に向ける」「軸を床に対して垂直に保つ」「脚をだす, 体重を移動する, ヒップを回転するを分ける」「常に床に力をかけ続ける」という4つの重要な技術要素が明らかになった. これらはクカラチャを実践する上での前提となる共通の技術要素であり, CHARMでは「常に」という注釈を付けることで対応したものの, このような指導の背後にある本質的な要素を適切に表現することは困難であった. この限界は単なる表現手法の問題ではなく, 技能伝承において何を伝えるべきかという本質的な課題に関わっている.\\
%  技能伝承を効果的に実現するためには, 伝承すべき知識の性質を理解し, それぞれに適した伝承方法を選択する必要がある. しかし, 技能に含まれる知識は多層的であり, その形式化の難しさも層によって異なる. 例えば具体的な動作手順は比較的形式化が容易である一方, 状況に応じた判断基準や, その技能分野が目指す本質的な価値は形式化が困難な場合が多い.\\
%  このような技能に含まれる知識の多層性と形式化可能性を体系的に理解するため, 本節では技能の階層的構造モデルを提案する. このモデルは技能を4つの層で捉え, 各層における知識の形式化状況に基づいて技能分野を分類することを可能にする. これにより, 各分野における技能伝承の課題をより明確に把握し, 適切な伝承方法を選択するための枠組みを提供する.\\

% \subsection{CHARMの適用における本質的な課題}
%  本研究の実験結果は, 社交ダンスにおける指導現場でのインタラクションを通じて新たな知識を抽出できることを示した一方で, CHARMによる知識表現に一定の限界があることも示唆した. この限界は単なる手法の問題ではなく, 技能伝承における知識構造の本質に関わる重要な示唆を提供している. \\
%  CHARMは本来, 行為間の目的達成関係を記述するためのモデルであり, 「ある目的を達成するために必要な行為の系列」を階層的に表現することを意図している. この枠組みは介護や製造業など, 目的と手段の関係が明確な分野では有効に機能してきた. しかし, 本研究で作成した社交ダンスのプロセス知識は, 結果的にCHARMの本来の意図から外れた使用方法となっていた. \\
%  例えば, クカラチャの動作において「右足に立つ」「左足を左へ送る」「左足を横にステップする」という行為の連鎖は, CHARMの階層構造として記述されている. しかし, これらの行為は上位の「レフトフットクカラチャ(男子)を行う」といった目的を達成するための手段という関係にはない. むしろ, これらは時間軸に沿って順序付けられた一連の動作の分解に過ぎない. さらに, 各動作に付与された「つま先を外側に向けて(常時)」「体重の移動(ボディとヒップが足の上まで)」といった要件は, 本来なら上位層で表現されるべき原理的な制約を, 個別の動作の属性として断片的に記述している. \\
%  このような問題の分析を通じて, 社交ダンスのような身体技能の分野では, 知識構造に異なる性質を持つ二つの側面が存在することが明らかになった. 一つは時系列的な動作の連鎖である. 例えば「クカラチャ」という動作は, 特定の身体の動きが, 決められた時間的順序で実行されることで初めて成立する. この時系列的な順序は, 目的達成のための論理的な必然性というよりも, その技能に固有の約束事として存在している. \\
%  もう一つは原理的な知識の階層構造である. 本研究で明らかになった「つま先を外側に向ける」「軸を床に対して垂直に保つ」「脚をだす, 体重を移動する, ヒップを回転するを分ける」「常に床に力をかけ続ける」という四つの技術要素は, 個々の動作の時系列的な順序とは独立に, その技能全体を支える原理として機能している. これらの要素は特定の時点での動作としてではなく, 継続的に維持すべき要件として存在する. \\
%  この分析は, CHARMの本来の強みをより適切に活かす方向性を示唆している. すなわち, CHARMは技能の原理的な側面, 特に「なぜその動作が必要か」「どのような効果を意図しているか」といった目的-手段の関係性を記述する枠組みとして活用すべきである. 一方で, 具体的な動作の時系列的な記述は, 別の表現形式を用いて単層的に記述することが適切である. \\
%  このような知識構造の二重性の認識は, 技能伝承をより体系的に理解するための新たな枠組みの必要性を示唆している. 以下では, まず技能分野全般を分類するための階層モデルを提示し, 次に動作の特性に基づく分類を行うことで, この二重性をより詳細に分析していく. \\

% ・暗黙知を暗黙知のまま伝えること. 形式知化する必要性.\\
% ・手続き知識の問題点.\\
% ・オンライン指導の文化の醸成.\\