\chapter{提案手法}
- 提案手法の全体像 -\\
 本研究の目的は熟練技能者が持つ知識を抽出し共有可能な形に構造化することである. 本研究では,西村らの研究\cite{Nishimura2017}で使用されたプロセス知識をベースにしながら, 技能伝承の現場における指導者と学習者の対話ログ,多量なオープンデータから構成された大規模言語モデル(LLM)を組み合わせることで,より実践的な知識の抽出手法を提案する. 具体的には技能の指導現場に着目する. まず, 技能継承の対象となる動作のベースとなる知識モデル(動作知識)を作成しておく. 次に, 指導現場における, 熟練技能者, すなわち指導者と学習者のやりとりを収集する. 次に指導現場で収集した情報をLLMが解析し既存の動作知識の改良点を生成し提案する. そしてLLMによる改良提案を受けた指導者が最終的に動作知識を改良する.\\

- 先行研究と提案手法の対応 -\\
 これらの要素を組み合わせることで, 先行研究の課題に対して複合的なアプローチが可能となる. まず, 指導現場での対話収集により, 未経験者の理解に関する課題に直接的にアプローチできる. 具体的には, 学習者からの質問や意見を通じて未経験者の視点をプロセス知識に反映できる. また, 学習者との対話の中で生まれる予期せぬ質問が, 熟練技能者が意識していなかった暗黙的な知識の抽出を促すきっかけとなり, これは知識記述の限界や記述漏れの課題の解決にもつながる.さらに, LLMの活用により, 収集した対話から新たな知見を抽出し, それを基に改良案を提示することで, 熟練技能者の新たな事例想起を促すことができる. これは記述漏れの抑制や知識記述の幅の拡大にも寄与する. 加えて, 指導現場という日常的な文脈の中で継続的に対話を収集することで, ワークショップ形式特有の時間的制約や事例想起の偏りといった課題も自然に解消される.このように, 提案手法は先行研究の個々の課題に対して, 複数の要素が相互に作用しながら解決を図るアプローチを取る.\\


- 提案手法の具体的なイメージ -\\
 本研究で提案する手法は以下の手順で実施される. まず, 先行研究と同様にベースとなるプロセス知識を作成する. 次に, 指導現場における指導者と学習者のやりとりを収集する. そして, 収集した情報をLLMが解析し, 既存のプロセス知識の改良点を生成・提案する. 最後に, LLMによる改良提案を受けて指導者がプロセス知識を改良する.\\